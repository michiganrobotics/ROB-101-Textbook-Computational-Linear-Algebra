The Robotics faculty want out of the Sputnik era of educating engineers, where we are forced to pound our students with four semesters of Calculus before we are able to engage them in any interesting engineering. We seek to prepare students for the era of Information, AI, Data, and of course, Robotics. While we believe our ideas for this Linear Algebra course will work for most engineering departments, we understand that the College of Engineering needs a skunkworks to test some of our more revolutionary suggestions before turning them loose on everyone. We are proud to serve that role.\\

% Engineering math education is stuck in the Sputnik era: we force you to do four semesters of calculus before you can do anything interesting in engineering. ROB 101 seeks to break through with new ideas. You will see how engineers are using mathematics and computing to solve large and important problems. You will still do drill problems to firm up concepts with teeny tiny problems with two or three variables, but you will also solve problems in the Julia programming language with hundreds of variables.

ROB 101 assumes a High School course in Algebra and no background in programming. With these entry requirements, we seek to open up mathematics and programming to everyone with the drive and skills to arrive at the University of Michigan's College of Engineering. From its inception in December 2019, the plan has always been to teach the course in a hybrid mode. We are being very intentional to design the course for inclusivity with a focus on making sure that one's zip code is not the best predictor of success. To do that, we are re-imagining the way mathematics is introduced to first-semester Y1 undergrads. We want to break the Sputnik era 4-semester calculus chain where AP credits are a huge predictor of success.\\

We will begin mathematics with Linear Algebra, the workhorse of modern autonomous systems, machine learning, and computer vision. We are integrating it with computation, and to make sure that students without access to high-end tools can be successful, all the programming will be run in a web browser through cloud services that are hidden from the student. If you can google at home or the library, then you can complete our programming assignments. Setting this up is a challenge, but we feel it is very important to have a platform that has equity in mind. Our plans for a hybrid mode of teaching are motivated by a desire to have students from Minority Serving Institutions join the course this fall.\\

The material in the course leaves out many traditional Linear Algebra topics, such as eigenvalues, eigenvectors, or how to compute determinants for matrices larger than $2 \times 2$. Sounds crazy! Good. The course focuses on solving systems of linear equations at scale, meaning hundreds or thousands of variables, and all the mathematics in the book should be implementable in HW sets in Julia. With that as a premise, we chose to focus on a few things in Linear Algebra that work well at scale, such as triangular systems of equations. This led to the conviction that LU Factorization should be introduced early and in an understandable manner, which was made possible by a magical video by Prof. Gilbert Strang (MIT). Once you understand that matrix multiplication $C= A \cdot B$ can be done by multiplying the columns of $A$ by the rows of $B$ and summing them up, the LU Factorization becomes completely transparent, allowing triangular matrices to rule the day. And once you focus on triangular structures, even linear independence and linear combinations become much more approachable. Of course, some basic geometry is good, and thus Gram-Schmidt is featured along with the QR Factorization. In between, least squared error solutions to linear equations and regression are treated along with other useful tools.\\

The book wraps up with a treatment of root finding for both scalar and vector nonlinear equations, and a users' view of optimization that highlights the role that the abstract function ``$\argmin$'' is playing in modern engineering. \\

Readers of the book will not find many applications of the juicy computational tools as they are treated in the HW sets, via jupyter notebooks, and in three amazing Projects. The first project leads students through the process of building a map for robot navigation from LiDAR data collected on the UofM North Campus Grove. The main Linear Algebra concept being explored is the transformation of points and collections of points in $\real^3$ under rigid body transformations. The second project is built around regression and will give students insight into the power of Machine Learning. The third project will focus on the control of a planar Segway using a simplified version of Model Predictive Control. Students will experience the excitement of balancing a challenging ubiquitous mobile platform, and while doing so, will learn about ODEs and their relation to iterative discrete-time models. \\

Finally, we thank Prof. Steven Boyd for making his Y1 Applied Linear Algebra course material open source (\url{http://vmls-book.stanford.edu/}). We have done the same (\url{https://tinyurl.com/sxk8n4u9}).\\

\textbf{Jessy Grizzle and Maani Ghaffari}
Ann Arbor, Michigan USA