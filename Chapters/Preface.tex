
This collection of course notes is dedicated to the group of students who were brave enough to take the pilot offering of ROB 101, Computational Linear Algebra in Fall 2020. \\

ROB 101 was conceived by Prof. Chad Jenkins as one part of a complete undergraduate curriculum in Robotics. Chad and I both thank Associate Dean for Undergraduate Education, Prof. Joanna Mirecki Millunchick, for her support in the offering of this course as ROB 101. \\

The following remarks are adapted from an Education Proposal led by Prof. Jenkins, Dr. Mark Guzdial, Ella Atkins, and myself.\\


A challenge for the current undergraduate curricular structure at the University of Michigan (and elsewhere) is that the best Robotics major is a quadruple major in ME, EE, CS, and Math with a minor in Psychology. The Robotics Institute at Michigan is poised to address this challenge in revolutionary ways as it evolves toward a department. The Robotics faculty are using the opportunity of a clean slate in the area of undergraduate robotics education-—and the absolute necessity to integrate learning across a wide range of traditional disciplines---to design a new curricular system where computation, hardware, and mathematics are on an equal footing in preparing a future innovation workforce.  \\

By integrating linear algebra, optimization, and computation in the first semester, students will experience mathematics as a means of making predictions and reasoning about experimental outcomes and robot designs. They will be prepared to encounter physics as a means to build mathematical models that describe the movement of objects, those with a palpable mass as well as electrons and waves, while grounding all of this in design. And computation? They will find it is the engine that drives discovery and the means of extracting information from data, allowing their machines to make decisions in real-time.\\

In addition to \textbf{ROB 101 ({\em Computational Linear Algebra})} in the first year, we are planning \textbf{ROB 102 ({\em Graph Search for Robotics and AI})}, which will show how computers can reason autonomously through graph search algorithms. The objective of the course is for students to implement a path planner for autonomous navigation with a given mobile robot at a known location in a known environment.  The course will build towards providing a broader conceptual foundation for modeling problems as graphs and inferring solutions through search. \\

In \textbf{ROB 103 ({\em Robotic Mechanisms})}, students will experience hands-on robotic systems design, build, and operation.  The objective of the course is for students, in teams, to build an omni-drive mobile robot that can be teleoperated, as a step towards the gateway course: ROB 204.  Students will learn to safely and efficiently operate modern shop tools such as 3D printers and laser cutters as well as traditional machining tools such as lathes, mills, and drill presses.  Students will learn basic electronic and power systems principles including safe battery management, wiring harness design and assembly, and signal measurement for test and debugging. Students will design and build their real-world mobile robot based on application requirements from conception with CAD software through manufacturing, assembly, and test.  Each student team will be given a ``kit'' of servos, sensors, and low-cost embedded processing components to use in their design.\\

The new first-year curriculum allows for major innovation in the second and third year curriculum, and the advancing of many graduate topics to much earlier places in Robotics education. We hope that you will follow our efforts in bringing this curriculum to the University of Michigan.\\

\textbf{Jessy Grizzle}\\
Fall Term, 2020