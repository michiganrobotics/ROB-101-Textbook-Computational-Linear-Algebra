\documentclass[letterpaper]{book}
\PassOptionsToPackage{usenames,dvipsnames,svgnames,table,xcdraw}{xcolor}
\PassOptionsToPackage{dvipsnames,svgnames,table,xcdraw}{enumitems}

\input{Defines/definesPackages}

\newcommand{\rb}[1]{\raisebox{1.5ex}{#1}}
 \newcommand{\trace}{\mathrm{trace}}



%\def\real{\mathbb{R}}
\newcommand{\real}{\mathbb R}
\newcommand{\nat}{\mathbb N}   
\newcommand{\whole}{\mathbb Z}    
\newcommand{\cp}{\mathbb C}    
\newcommand{\rat}{\mathbb Q} 
\newcommand{\im}{\mathbb i}
\newcommand{\impart}[1]{\mathrm{imag} ( #1 )}
\newcommand{\repart}[1]{\mathrm{real} ( #1 )}
%\newcommand{\angle}[1]{\mathrm{angle} (#1 )}

%\def\real{\mathbb{R}}
\def\reals{\mathbb{R}} %????

\newcommand{\ds}{\displaystyle}
\newcommand{\mf}[2]{\frac{\ds #1}{\ds #2}}
\newcommand{\book}[2]{{Boyd, Page~#1, }{Prob.~#2}}
\newcommand{\Kuttler}[1]{Kuttler 2017, A First Course in Linear Algebra, Page(s)~#1}
\newcommand{\spanof}[1]{\mathrm{span} \{ #1 \}}
\newcommand{\colspanof}[1]{\mathrm{col~span} \{ #1 \}}
\newcommand{\nullspace}[0]{\mathrm{null}}
\newcommand{\nullity}[0]{\mathrm{nullity}}
\newcommand{\range}[0]{\mathrm{range}}
\newcommand{\cost}[0]{\mathrm{f}}
\newcommand{\rank}[0]{\mathrm{rank}}
\newcommand{\diag}[0]{\mathrm{diag}}

%%%%%%%%%%%%%%%%%%%
% Bruce added below
%%%%%%%%%%%%%%%%%%%
\newcommand{\transpose}{\mathsf{T}}
\DeclareDocumentCommand{\zeros}{ O{} }{\textbf{0}_{#1}}
\newcommand\inv[1]{#1\raisebox{1.15ex}{$\scriptscriptstyle-\!1$}}
\newcommand{\Exp}{\mathrm{Exp}}
\newcommand{\Log}{\mathrm{Log}}
\newtheorem{remark}{Remark}
%%%%%%%%%%%%%%%%%%%
% Bruce added above
%%%%%%%%%%%%%%%%%%%



%\newcommand{\dim}[0]{\mathrm{dim}}
 \newcommand{\cov}{\mathrm{cov}}
 \newcommand{\E}{\mathcal{E}}
\parindent 0pt

\newcommand{\notimplies}{%
  \mathrel{{\ooalign{\hidewidth$\not\phantom{=}$\hidewidth\cr$\implies$}}}}


\DeclareMathOperator*{\argmin}{arg\,min}
\DeclareMathOperator*{\argmax}{arg\,max}

\newcommand{\hilight}[1]{\colorbox{yellow}{#1}} % to use: \hl{this is some highlighted text}
\newcommand{\red}[1]{{\textcolor{red}{#1}}}
\newcommand{\blue}[1]{{\textcolor{blue}{#1}}}
\newcommand{\jwg}[1]{[{\textbf{\textcolor{red}{JWG: #1}}}]}
\newcommand{\mgj}[1]{[{\textbf{\textcolor{teal}{Maani: #1}}}]}
\newcommand{\bh}[1]{[{\textbf{\textcolor{blue}{Bruce: #1}}}]}


\newcommand\RED{\color{red}}
\newcommand\BLUE{\color{blue}}

\definecolor{note}{rgb}{0.3,0.7,0.25}
\definecolor{rephase}{rgb}{0.15,0.7,0.15}
\definecolor{bag}{rgb}{0.6,0.6,0.2}

 \newcommand\x{\times}
\newcommand\bigzero{\makebox(0,0){\text{\huge0}}}
\newcommand*{\bord}{\multicolumn{1}{c|}{}}
\newcommand*{\bordl}{\multicolumn{1}{|c}{}}

\newcommand*{\Qed}{\hfill $\blacksquare$}

\newcommand\doverline[1]{\ThisStyle{%
  \setbox0=\hbox{$\SavedStyle\overline{#1}$}%
  \ht0=\dimexpr\ht0-.15ex\relax% CHANGE .15 TO AFFECT SPACING
  \overline{\copy0}%
}}

\newcommand{\rob}{\textbb{ROB 101}}

% \newcommand*\circled[1]{\tikz[baseline=(char.base)]{
%     \node[shape=circle, draw, inner sep=1pt, 
%         minimum height={\f@size*1.6},] (char) {\vphantom{WAH1g}#1};}}
\makeatother







\usepackage{listings}
\usepackage{xcolor}

\definecolor{dkgreen}{rgb}{0,0.6,0}
\definecolor{gray}{rgb}{0.5,0.5,0.5}
\definecolor{mauve}{rgb}{0.58,0,0.82}
\definecolor{backcolour}{rgb}{0.98,0.92,0.97}
\definecolor{codegray}{rgb}{0.5,0.5,0.5}

%%
%% Julia definition (c) 2014 Jubobs
%%
\lstdefinelanguage{Julia}%
  {morekeywords={abstract,break,case,catch,const,continue,do,else,elseif,%
      end,export,false,for,function,immutable,import,importall,if,in,%
      macro,module,otherwise,quote,return,switch,true,try,type,typealias,%
      using,while},%
   sensitive=true,%
   alsoother={$},%
   morecomment=[l]\#,%
   morecomment=[n]{\#=}{=\#},%
   morestring=[s]{"""}{"""},%
   morestring=[m]{"}{"},%
}[keywords,comments,strings]%

\lstdefinestyle{mystyle}{
frame=tb,
  language=Julia,
  aboveskip=3mm,
  belowskip=3mm,
  showstringspaces=false,
  columns=flexible,
  basicstyle=\ttfamily,
  backgroundcolor=\color{backcolour},
  numberstyle=\tiny{\bf \color{black}},
  keywordstyle={\bf \color{blue}},
  commentstyle=\color{dkgreen},
  identifierstyle={\bf \color{black}},
  stringstyle=\color{mauve},
  breakatwhitespace=false,         
  breaklines=true,                 
  captionpos=b,                    
  keepspaces=true,                 
  numbers=left,                    
  numbersep=5pt,                  
  showspaces=false,                
  showstringspaces=false,
  showtabs=false,                  
  tabsize=2
}

\lstset{style=mystyle}

    % commentstyle=\itshape\color{purple!40!black},
    % identifierstyle=\color{blue},
    % backgroundcolor=\color{gray!10!white}
%\parindent

\makeatletter
\renewcommand{\frontmatter}{\cleardoublepage\@mainmatterfalse}
\renewcommand{\mainmatter}{\cleardoublepage\@mainmattertrue}
\makeatother

\title{Notes for Computational Linear Algebra\thanks{Inspired by Prof. Chad Jenkins}}
\date{}
\author{Jessy Grizzle\\ Robotics Institute, University of Michigan, Ann Arbor
\and Content contributed by Maani, Tribhi, Miley, Madhav, Kira, Michael, and Bruce}

%\doublespacing



\begin{document}
%\maketitle
\newtheorem{example}{Example}
\numberwithin{example}{chapter}





\section{Lecture 03}

\begin{tcolorbox}[sharp corners, colback=green!30, colframe=green!80!blue, title=\textbf{\Large Enough Facts about the Determinant to Get Us Going}]
\begin{center}
\begin{minipage}{.95\textwidth}


\begin{enumerate}[label={\bf Fact~~}{\bf \arabic*}]
\setlength{\itemsep}{.5cm}
\Large

\item \label{item:determinantFact0}  The determinant of a square matrix $A$ is a real number, denoted $\det(A)$.

    \item \label{item:determinantFact1} A square system of linear equations (i.e.,  $n$ equations and $n$ unknowns), 
    $Ax=b,$
    has a unique solution $x$ for any $n\times 1$ vector $b$ if, and only if, $\det(A) \neq 0.$
    
     \item When $\det(A)=0$, the set of linear equations $Ax=b$ may have either no solution or an infinite number of solutions. To determine which case applies (and to be clear, only one case can apply), depends on how ``$b$ is related to $A$'', which is another way of saying that we will have to address this later in the course. 
     
     \item The determinant of the $1 \times 1$ matrix $A=[a]$ is $\det(A):=a$.
     
    \item The determinant of the $2 \times 2$ square matrix 
    $A=\left[\begin{array}{cc} a & b \\ c & d \end{array} \right]$ is
  $\det(A):= a d - bc.$

 
\end{enumerate}

\end{minipage}
\end{center}
\end{tcolorbox}

\newpage
\section{Lecture 14}

\begin{tcolorbox}[title={\bf \Large Properties of the Dot (or Inner) Product}]
\Large 
For vectors $v_1, v_2, v_3 \in \real^n$ and real numbers $\alpha_1$ and $\alpha_2$, the following properties hold:

\vspace*{1cm}

\begin{itemize}
\setlength{\itemsep}{.3cm}
    \item $v_1 \bullet v_2 = v_2 \bullet v_1$ 
    \begin{align*}
       \textbf{Because:}~~ v_1 \bullet v_2 &: = v_1^\top v_2   \\
         &~= (v_1^\top v_2)^\top ~~\text{a real number}\\
        &~=v_2^\top v_1\\
         &~ =:  v_2 \bullet v_1 
    \end{align*}

    \item $v_1 \bullet (v_2 + v_3) = v_1 \bullet v_2 + v_1 \bullet v_3$ 
    \begin{align*}
       \textbf{Because:}~~ v_1 \bullet (v_2 + v_3)& : = v_1^\top (v_2 + v_3)  \\
        &~=v_1^\top v_2 + v_1^\top v_3 \\
        &~=: v_1 \bullet v_2 + v_1 \bullet v_3
    \end{align*}
        \item $(v_1 +v_2) \bullet  v_3 = v_1 \bullet v_3+ v_2 \bullet v_3$ 
    \begin{align*}
       \textbf{Because:}~~ (v_1 + v_2)\bulletv_3 & : = (v_1 + v_2)^\top v_3  \\
        &~=(v_1^\top  + v_2^\top) v_3 \\
         &~=v_1^\top v_3   + v_2^\top v_3 \\
         &~ =:  v_1 \bullet v_3 + v_2 \bullet v_3
    \end{align*}
                \item $(\alpha_1 v_1) \bullet  v_2 = \alpha_1 (v_1 \bullet v_2) = v_1 \bullet (\alpha v_2)$ 
    \begin{align*}
       \textbf{Because:}~~ (\alpha_1 v_1)\bullet v_2 & : = (\alpha_1 v_1)^\top v_2  \\
        &~=(\alpha_1 v_1^\top ) v_2 \\
       &~=\alpha_1 (v_1^\top v_2 )  \\
         &~ =:  \alpha_1 (v_1 \bullet v_2) \\
         & \text{\bf and} \\
         &~=  \alpha_1 (v_1^\top v_2 )  \\
         &~=  v_1^\top( \alpha_1 v_2 )   \\
         &~ = : v_1 \bullet (\alpha v_2)
    \end{align*}
    %         \item $(\alpha_1 v_1 + \alpha_2 v_2) \bullet  v_3 = \alpha_1 v_1 \bullet v_3+ \alpha_2 v_2 \bullet v_3$ 
    % \begin{align*}
    %   \textbf{Because:}~~ (\alpha_1 v_1 + \alpha_2 v_2)\bullet v_3& : = \alpha_1 v_1 + \alpha_2 v_2)^\top v_3  \\
    %     &~=(\alpha_1 v_1^\top  + \alpha_2 v_2^\top) v_3 \\
    %      = &\alpha_1 v_1^\top v_3   + \alpha_2 v_2^\top v_3 \\
    %     &~=: \alpha_1 v_1 \bullet v_3 + \alpha_2 v_2 \bullet v_3
    % \end{align*}
\end{itemize}

\end{tcolorbox}

\section{FAMU Talk}

\Large

\begin{equation} \label{eq:Desired_Footplacement}
    p^{x~{\rm des}}_{\rm \bf sw \to CoM}(T_k^-,t) :=\frac{L^{y ~{\rm des}} - \cosh(\ell T)\widehat{L}(T_k^-,t)}{mH \ell \sinh(\ell T)}.
\end{equation}

\vspace*{.5cm}

\noindent \textbf{Remark:} Instead of the deadbeat control \eqref{eq:Desired_Footplacement}, it is possible to asymptotically approach a desired value of $L^{\rm des}$ with the control law
\begin{equation} \label{eq:Desired_FootplacementAsymptotic}
\begin{aligned}
    p^{x~{\rm des}}_{\rm \bf sw \to CoM}(T_k^-,t) &:=\frac{1 - \alpha }{mH \ell \sinh(\ell T)}L^{y ~{\rm des}} \\
   & +  \frac{\alpha - \cosh(\ell T)}{mH \ell \sinh(\ell T)} \widehat{L}(T_k^-,t),
    \end{aligned}
\end{equation}
which achieves
\begin{equation}
( L^{y~{\rm des}}-\widehat{L}(T_{k+1}^-,t))   = \alpha (L^{y~{\rm des}}-\widehat{L}(T_{k}^-,t))
\end{equation}
% \begin{equation}
% ( L^{y~{\rm des}}(T_{k+1}^-)-\widehat{L}(T_{k+1}^-,t))   = \alpha (L^{y~{\rm des}}(T_{k+1}^-)-\widehat{L}(T_{k}^-,t))
% \end{equation}
for $\alpha \in [0,1)$. Hence, for $\alpha = 0$, \eqref{eq:Desired_FootplacementAsymptotic} reduces to \eqref{eq:Desired_Footplacement}.

\newpage

\section{Eigen Stuff}
\large

\textbf{Example 10.6}  \label{ex:Chap10Eigen01} Let $A$ be the $2 \times 2$ real matrix
 $A=\left[\begin{array}{rr}
    1 & 2\\
    3 & 2
    \end{array}\right].$
Determine, if any, its eigenvalues and eigenvectors. 

\vspace*{.2cm}

\textbf{Solution:} To find eigenvalues, we need to solve 
$$\det(\lambda I-A)= \left| \begin{array}{cc}
    \lambda-1 & -2\\
   -3 &\lambda -2
    \end{array} \right| =(\lambda-1)(\lambda-2)-6=\lambda^2- 3 \lambda -4=0.$$
    We compute the discriminant of this quadratic equation and we find
    $$b^2-4ac = 9 +16 =25 >0,$$
    and therefore there are two real solutions. We compute them with the quadratic formula to be $\lambda_1=-1$ and $\lambda_2=4$.\\
    
   To determine an eigenvector associated with $ \lambda_1=-1$, we need to find $v_1\in \real^2$ such that 
   \begin{align*}
       (A-\lambda_1 I_2) v_1 & = 0_{2 \times 1}\\
        & \Updownarrow \\
   \left( \left[\begin{array}{rr}
    1 & 2\\
    3 & 2
    \end{array}\right] - (-1)   \left[\begin{array}{rr}
    1 & 0\\
    0& 1
    \end{array}\right]\right) \left[\begin{array}{r}
   v_{1a} \\
  v_{1b}
    \end{array}\right] & = \left[\begin{array}{r}
   0 \\
  0
    \end{array}\right]\\
     & \Updownarrow \\
      \left[\begin{array}{rr}
    2 & 2\\
    3 & 3
    \end{array}\right]  \left[\begin{array}{r}
   v_{1a} \\
  v_{1b}
    \end{array}\right] & = \left[\begin{array}{r}
   0 \\
  0
    \end{array}\right]\\
    & \Updownarrow \\
   \left[\begin{array}{r}
   v_{1a} \\
  v_{1b}
    \end{array}\right] & = \alpha_1  \left[\begin{array}{r}
   1 \\
  -1
    \end{array}\right], ~\alpha_1 \neq 0.
   \end{align*}
  
   Similarly, to determine an eigenvector associated with $ \lambda_2=4$, we need to find $v_2\in \real^2$ such that 
   \begin{align*}
       (A-\lambda_2 I_2) v_2 & = 0_{2 \times 1}\\
        & \Updownarrow \\
   \left( \left[\begin{array}{rr}
    1 & 2\\
    3 & 2
    \end{array}\right] - (4)   \left[\begin{array}{rr}
    1 & 0\\
    0& 1
    \end{array}\right]\right) \left[\begin{array}{r}
   v_{2a} \\
  v_{2b}
    \end{array}\right] & = \left[\begin{array}{r}
   0 \\
  0
    \end{array}\right]\\
     & \Updownarrow \\
      \left[\begin{array}{rr}
    -3 & 2\\
    3 & -2
    \end{array}\right]  \left[\begin{array}{r}
   v_{2a} \\
  v_{2b}
    \end{array}\right] & = \left[\begin{array}{r}
   0 \\
  0
    \end{array}\right]\\
    & \Updownarrow \\
   \left[\begin{array}{r}
   v_{2a} \\
  v_{2b}
    \end{array}\right] & = \alpha_2  \left[\begin{array}{r}
   2\\
 3
    \end{array}\right], ~\alpha_2 \neq 0.
   \end{align*}
    

\Qed

\newpage
\Large

\begin{tcolorbox}[title=\textbf{\Large Finding Eigenvectors and Eigenvalues with Julia}]
Beyond $2 \times 2$ matrices, we do not compute eigenvalues or eigenvectors by hand! In ROB 101, we use Julia!

\begin{lstlisting}[language=Julia,style=mystyle]
Random.seed!(876543212345678);
B=randn(4,4)
A=B'*B # symmetric matrices have real eigenvalues

E=eigen(A)
@show E.values
E.vectors
\end{lstlisting} 
\end{tcolorbox}
\begin{verbatim}
E.values = [0.06287200462929299, 0.6813033999332612, 
            2.9738855645273268, 4.4839915456638]

4 x 4  Matrix{Float64}:
  0.339074   0.0385456  -0.71411   -0.61122
  0.551488   0.528452   -0.226828   0.604275
 -0.191731   0.824533    0.318536  -0.426521
  0.737651  -0.19849     0.58063   -0.281676
\end{verbatim}

\newpage


\vspace*{0.5cm}
\begin{tcolorbox}[sharp corners, colback=green!30, colframe=green!80!blue, title=\textbf{\Large When the Eigenvalues are Real and Distinct, the Eigenvectors form a Basis of $\real^n$}]
Let $A$ be an $n \times n$ matrix with real coefficients. If the eigenvalues $\{ \lambda_1,\ldots, \lambda_n \}$ are real and \textbf{distinct}, that is, $\lambda_i \neq \lambda_j $ for all $1 \le i \neq j \le n$, then the eigenvectors $\{ v_1,\ldots,v_n \}$ are real and provide a basis of $\real^n.$\\

Once again, the full story is given in Appendices~\ref{sec:CaseEigenvectors} and~\ref{sec:EigenStuff}. An interesting tidbit is that symmetric matrices always have real eigenvalues. Moreover, their eigenvectors can always be selected to form an orthogonal matrix; see Appendix~\ref{sec:RealSymmetricMatrices}.
\end{tcolorbox}

\vspace*{0.5cm}

\begin{example}
\label{ex:BasisEigenvectors}
Using Julia, find the eigenvalues and eigenvectors of the $3 \times 3$ (symmetric) matrix below. Furthermore, determine if the eigenvectors form a basis of $\real^3$.

\begin{equation}
A:=\left[
\begin{array}{rrr}
2.2216 & 1.6798 & -0.2670 \\
1.6798 & 0.8457 & -0.1651 \\
-0.2670 & -0.1651 & 0.6391 \\
\end{array}
\right].
\end{equation}
\end{example}

\textbf{Solution:} 
\begin{lstlisting}[language=Julia,style=mystyle]
E=eigen(A)
E.values
E.vectors
det(E.vectors)
\end{lstlisting}
Using Julia, we compute that 
\begin{equation}
\left[
\begin{array}{rrr}
\lambda_1&
\lambda_3 &
\lambda_3
\end{array}
\right] =\left[
\begin{array}{rrr}
-0.2817 &
0.6034 &
3.3848 
\end{array}
\right]
\end{equation}
\text{ and } 
\begin{equation}\left[
\begin{array}{rrr}
v_1&
v_3 &
v_3
\end{array}
\right] =
\left[
\begin{array}{rrr}
0.5581 & 0.0870 & -0.8252 \\
-0.8297 & 0.0741 & -0.5533 \\
0.0131 & 0.9934 & 0.1135 \\
\end{array}
\right].
\end{equation}
Because the eigenvalues are distinct, we know that set $\{v_1, v_2, v_3 \}$ forms a basis of $\real^3$. To double check this, we determine that $\det(E.vectors)=1 \neq 0$ so that indeed, the eigenvectors are linearly independent and hence form a basis. 
\Qed

\newpage

\begin{tcolorbox}[sharp corners, colback=green!30, colframe=green!80!blue, title=\textcolor{red}{\Large \bf Utility of Eigenvalues and Eigenvectors:} \textbf{\Large They Explain how a Square Matrix acts on a Vector}]
Let $A$ be an $n \times n$ real matrix with real eigenvalues $\{ \lambda_1,\ldots, \lambda_n \}$ that are \textbf{distinct}, that is, $\lambda_i \neq \lambda_j $ for all $1 \le i \neq j \le n$. It then follows that the eigenvectors $\{ v_1,\ldots,v_n \}$ provide a basis for $\real^n.$ Let $x\in \real^n$ be arbitrary and write it as a linear combination of the basis of eigenvectors
\begin{equation}
    \label{eq:xExpandedEigenBasis}
    x = \alpha_ 1 v_1 + \alpha_2 v_2 + \cdots + \alpha_n v_n.
\end{equation}
Then because $Av_i = \lambda_i v_i$,
\begin{equation}
    \label{eq:xExpandedEigenBasisA}
   A x = \alpha_ 1 \lambda_1 v_1 + \alpha_2 \lambda_2 v_2 + \cdots +\alpha_n  \lambda_n v_n.
\end{equation}
If we apply $A$ to both sides of \eqref{eq:xExpandedEigenBasisA}, we obtain
\begin{equation}
    \label{eq:xExpandedEigenBasisA2}
    \begin{aligned}
       A^2 x &= \alpha_ 1 \lambda_1 A v_1 + \alpha_2 \lambda_2 A v_2 + \cdots +\alpha_n  \lambda_n Av_n \\
       &= \alpha_ 1 (\lambda_1)^2 v_1 + \alpha_2 (\lambda_2)^2  v_2 + \cdots +\alpha_n  (\lambda_n)^2 v_n,
    \end{aligned}
\end{equation}
where $A^2:= A \cdot A$ and we have used again, $A v_i = \lambda_i v$. 
Moreover, using this fact iteratively yields that, for all $k \ge 2$,
\begin{equation}
    \label{eq:xExpandedEigenBasisB}
   A^k x = \alpha_ 1 (\lambda_1)^k v_1 + \alpha_2 (\lambda_2)^k v_2 + \cdots +\alpha_n  (\lambda_n)^k v_n.
\end{equation}

\end{tcolorbox}
\vspace*{.3cm}

\textbf{Remarks:} 
\begin{itemize}
    \item Equation \eqref{eq:xExpandedEigenBasisB} for $k=1$ says that if we write a vector $x$ in the coordinates provided by the ``eigen-basis'' of a matrix, then how the matrix acts on the vector is very transparent: it simply scales each component by the corresponding eigenvalue. When a matrix has real eigenvalues, what we perceive as the matrix ``rotating a vector'' in the natural basis vectors $\{e_1, e_2, \ldots, e_n \}$ is an illusion; what is really happening is that the matrix is expanding, contracting, or leaving the same length individual components of the vector along various directions determined by the matrix's eigenvectors. 
    \item The above property is exploited heavily in the design of feedback control systems. 
    \item Equation \eqref{eq:xExpandedEigenBasisB} explains the phenomenon in Fig.~\ref{fig:SquishingSpace}, because 
    $$ \begin{cases} 
    |\lambda_i| < 1 & \iff  \lim_{k \to \infty}  ||(\lambda_i)^k v_i||= \lim_{k \to \infty}  |\lambda_i|^k  ||v_i||=0\\  
    |\lambda_i| >1 & \iff  \lim_{k \to \infty}  ||(\lambda_i)^k v_i||= \lim_{k \to \infty}  |\lambda_i|^k  ||v_i||= \infty \\
    |\lambda_i|= 1 & \iff || (\lambda_i)^k v_i|| = |\lambda_i|^k  || v_i||= ||v_i||, k\ge 0 
    \end{cases} 
    $$

\end{itemize}



\end{document}